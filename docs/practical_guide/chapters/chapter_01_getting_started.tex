\section{Introduction}

\texttt{hpfracc} is a high-performance framework for fractional calculus, designed to bridge the gap between rigorous mathematical modeling and scalable computational implementation. It supports a wide range of fractional operators (Caputo, Riemann-Liouville, Riesz, etc.) and integrates seamlessly with modern deep learning workflows.

\section{Installation}
To install \texttt{hpfracc}, use pip:

\begin{lstlisting}[language=bash]
pip install hpfracc
\end{lstlisting}

Ensure you have the necessary backends installed (PyTorch, JAX, or Numba) depending on your performance requirements.

\section{First Steps: Calculating a Derivative}
The core of \texttt{hpfracc} is intuitive. Here is a simple example calculating the fractional derivative of $f(t) = t^2$.

\begin{lstlisting}[language=Python]
import numpy as np
from hpfracc.core.derivatives import CaputoDerivative

# Define grid and function
t = np.linspace(0.01, 5, 100)
f = t**2

# Initialize operator with alpha=0.5
operator = CaputoDerivative(order=0.5)

# Compute
result = operator.compute(f, t)
\end{lstlisting}

\section{Production Readiness}
Version 3.2.0 introduces several production-ready features:
\begin{itemize}
    \item **Intelligent Backend Selection**: Automatically selects the fastest backend (Numba/NumPy for small arrays, JAX/Torch for large tensors).
    \item **Standardized API**: Consistent `compute(function, points)` interface across all operators.
    \item **Validations**: rigorously validated against analytical solutions for standard functions.
\end{itemize}
