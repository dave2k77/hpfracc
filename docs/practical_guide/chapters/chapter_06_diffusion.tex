\section{Introduction}

Anomalous diffusion is a transport process where the mean squared displacement (MSD) scales as a non-linear power of time: $\langle x^2(t)\rangle \propto t^\alpha$. This contrasts with normal (Fickian) diffusion where $\alpha=1$.

\section{Simulation with Fractional Brownian Motion}

\texttt{hpfracc} provides tools to simulate Fractional Brownian Motion (fBm), a continuous Gaussian process with stationary increments that exhibits anomalous diffusion behavior.

\begin{lstlisting}[language=Python]
import numpy as np

def generate_fbm(n_samples, hurst):
    # Spectral synthesis method
    beta = 2 * hurst + 1
    freqs = np.fft.rfftfreq(n_samples * 2)
    magnitude = np.zeros_like(freqs)
    magnitude[1:] = freqs[1:] ** (-beta / 2.0)
    phase = np.random.uniform(0, 2*np.pi, size=len(freqs))
    fgn = np.fft.irfft(magnitude * np.exp(1j * phase))
    return np.cumsum(fgn[:n_samples])
\end{lstlisting}

\section{Analysis}
By characterizing the Hurst exponent ($H = \alpha/2$), researchers can identify the nature of the diffusion:
\begin{itemize}
    \item $H < 0.5$: Subdiffusion (antipersistent, $\alpha < 1$).
    \item $H = 0.5$: Normal diffusion ($\alpha = 1$).
    \item $H > 0.5$: Superdiffusion (persistent, $\alpha > 1$).
\end{itemize}
