\section{The Fractional MLOps Lifecycle}

Deploying fractional models requires specialized management of metadata (e.g., fractional order \(\alpha\), solver methods). \texttt{hpfracc} includes a built-in MLOps system inspired by MLflow but tailored for scientific machine learning.

\section{Key Components}

\begin{itemize}
    \item \textbf{ModelRegistry}: A local database tracking model versions, hyperparameters, and artifacts.
    \item \textbf{DevelopmentWorkflow}: Manages experiment tracking and initial model registration.
    \item \textbf{ProductionWorkflow}: Enforces quality gates before models can be promoted to 'Production' status.
\end{itemize}

\section{Workflow Example}

\subsection{1. Registration (Development)}
\begin{lstlisting}[language=Python]
dev_workflow = DevelopmentWorkflow()
model_id = dev_workflow.register_development_model(
    model=model,
    name="alpha_forecaster",
    version="0.1.0",
    fractional_order=0.6,
    hyperparameters={"hidden_size": 32}
)
\end{lstlisting}

\subsection{2. Validation}
Quality gates ensure robustness.
\begin{lstlisting}[language=Python]
validation_results = dev_workflow.validate_development_model(
    model_id=model_id,
    test_data=val_data,
    test_labels=val_labels
)
\end{lstlisting}

\subsection{3. Promotion}
If validation passes, promote to production.
\begin{lstlisting}[language=Python]
prod_workflow = ProductionWorkflow(registry=dev_workflow.registry)
prod_workflow.promote_to_production(model_id=model_id, version="0.1.0")
\end{lstlisting}

This system ensures reproducibility in scientific experiments, where parameter precise tracking is critical.
