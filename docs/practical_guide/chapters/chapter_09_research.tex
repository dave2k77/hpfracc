\section{Research Workflow}

When applying fractional calculus to novel research, reproducibility and performance are paramount. \texttt{hpfracc} facilitates this via `GPUProfiler` and `VarianceMonitor`.

\section{Case Study: Viscoelastic Materials}

Modeling stress relaxation in polymers often uses the fractional Kelvin-Voigt model:
\begin{equation}
\sigma(t) = E \varepsilon(t) + \eta D^\alpha \varepsilon(t)
\end{equation}

\begin{lstlisting}[language=Python]
from hpfracc.core.derivatives import CaputoDerivative

def stress_response(strain, t, E, eta, alpha):
    # Elastic part
    elastic = E * strain
    
    # Viscous part (fractional derivative)
    caputo = CaputoDerivative(order=alpha)
    viscous = eta * caputo.compute(strain, t)
    
    return elastic + viscous
\end{lstlisting}

\section{GPU Profiling}

For large-scale simulations (e.g., FEM integration), ensuring GPU utilization is rigorous.

\begin{lstlisting}[language=Python]
from hpfracc.ml.gpu_optimization import GPUProfiler

profiler = GPUProfiler()
with profiler.profile():
    # Massive computation
    result = compute_material_response(mesh_data)
\end{lstlisting}
